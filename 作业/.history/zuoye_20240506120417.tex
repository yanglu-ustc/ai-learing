\documentclass{article}
%使用中文
\usepackage{ctex}
%设置页面
\usepackage{geometry}
%三线格宏包
\usepackage{booktabs}
%插入图片
\usepackage{graphicx}
%如果希望避免浮动体跨过 \section,可以使用 placeins 宏包
\usepackage[section]{placeins}
\geometry{a4paper,left=2cm,right=2cm,top=1cm,bottom=1cm}
\title{单摆实验报告}
\author{姓名:卢杨 \quad 学号:PB23000043 \quad 班级:少院2023级2班}
\begin{document}
\maketitle
\section{实验目的}
    1. 利用单摆的周期公式
    
    \[ T=2\pi\sqrt{\frac{l}{g}}\]
    
    测定当地的重力加速度 $g$。其中 $T$ 是单摆的周期,$l$ 是单摆的摆长。

    2.调节单摆的摆长,测量摆长 $l$ 与周期 $T$ 之间的关系。拟合出对应图像,测出$g$.

    3. 研究大摆角($>5^{\circ }$)条件下单摆的运动规律。对运动轨迹数据进行拟合分析
\section{实验仪器}
    钢卷尺、游标卡尺(vernier calipers)、千分尺(micrometer)、电子秒表(stopwatch)、单摆(带标尺、平面镜;摆线长度可调,其可调的上限约为 100 cm)、智能手机
        
    \begin{figure}[!hbtp]
        \centering
        \includegraphics{./images/装置.png}
        \caption{实验仪器}
    \end{figure} 

    仪器注意事项:

    各测量仪器的最大允差如下:
    游标卡尺$\bigtriangleup_{v}\approx 0.002 cm$;千分尺$\bigtriangleup_{m}\approx 0.001 cm$;秒表$\bigtriangleup_{s}\approx 0.05 s$;
    用钢卷尺测量单摆摆长时难以将被测物两端与测量仪器的刻线对齐,作为保守估计,
    一般可取不确定度$\bigtriangleup_{r}\approx 0.2 cm$

    根据统计分析,实验人员开启或停止秒表的反应时间为0.1 s 左右,所以实
    验人员测量时间的不确定度近似为$\bigtriangleup_{person}\approx 0.2s$

    开始实验前,应调节螺栓使立柱竖直,并调节标尺高度,使其上沿中点距
    悬挂点 50 cm
\section{实验原理}
    
    单摆的周期公式为:
    \[ T=2\pi \sqrt{\frac{l}{g}[1+\frac{d^{2}}{20l^{2}}-\frac{m_{0}}{12m}(1+\frac{d}{2l}+\frac{m_{0}}{m})+\frac{\rho _{0}}{2\rho }+\frac{\theta ^{2}}{16}]} \]
    式中 $T$ 是单摆的周期,$l$ 、$m_{0}$是单摆摆线的长度和质量, $d$ 、$m$ 、$\rho $ 是摆球的直
    径、质量和密度,$\rho  _{0}$ 是空气密度,$\theta $ 是摆角。
    
    一般情况下,摆球几何形状、摆线的质量、空气浮力、摆角($\theta $<5°)对T 的修正都小于 $10^{-3}$
    。若实验精度要求在 $10^{-3}$以内,则这些修正项都可以忽略不计
    
    我们的实验要求是不确定度要优于$1\%$,也就是说这些因素中只有第一项需要考虑
    ,我们进行一级简化后得到公式

    \[ T=2\pi\sqrt{\frac{l}{g}}. \]

    进而得到公式:

    \[g=\frac{4\pi^{2}}{T^{2}}l\]

    我们只需要对这个公式进行讨论即可
    
\section{实验方案设计}
    \subsection{单摆法测量重力加速度的不确定度合成公式}
        \[g=\frac{4\pi^{2}}{T^{2}}l\]

        求偏导有:

        \[\frac{\partial g}{\partial l}=\frac{4\pi^{2}}{T^{2}}=\frac{g}{l}\]
        \[\frac{\partial g}{\partial T}=-2\frac{4\pi^{2}}{T^{3}}l=-2\frac{g}{T}\]

        得到不确定度公式为:

        \[U_{g}=\sqrt{(\frac{g}{l}U_{l})^2+(-2\frac{g}{T}U_{T})^2}\]
        \[\frac{U_{g}}{g}=\sqrt{(\frac{U_{l}}{l})^2+4(\frac{U_{T}}{T})^2}\]

        
    \subsection{ 对测量周期个数的探讨}

        实验要求是不确定度要优于$1\%$,也就是说$\frac{U_{g}}{g}<1\%$

        我们对此进行估计:

        \[\bigtriangleup_{B}=\sqrt{\bigtriangleup_{v}^2+\bigtriangleup_{m}^2+\bigtriangleup_{r}^2}\]

        $\bigtriangleup_{v}\approx 0.002 cm$
        $\bigtriangleup_{m}\approx 0.001 cm$
        $\bigtriangleup_{r}\approx 0.2 cm$

        如果一个分量小于另一个分量的三分之一,可以忽略较小的分量,故我们只考虑钢尺的不确定度:
        
        \[U_{l}=U_{r}\]

        同样

        \[\bigtriangleup_{person}>3\bigtriangleup_{s}\]
        \[U_{T}=U_{person}\]

        我们设置的初始长度是$70cm$

        \[(\frac{U_{l}}{l})^2\approx (\frac{0.2}{70})^2 ,\quad \frac{U_{g}}{g}<1\%\]
        \[\frac{U_{T}}{T}=\frac{1}{2}\sqrt{(\frac{U_{g}}{g})^2-(\frac{U_{l}}{l})^2}\approx 4.79\times 10^{-3}\]

        我们取重力加速度$g=9.8m/s^{2}$,带入计算,估计一下来回周期:

        \[T_{0}=2\pi\sqrt{\frac{l}{g}}\approx 1.679s\]

        如果我们一个来回测量一次

        \[\frac{U_{T}}{T}=\frac{0.2}{1.679}\approx 0.119 \gg 4.79\times 10^{-3}\]

        假设$N$个来回测量一次测量

        \[\frac{U_{T}}{NT}=4.79\times 10^{-3}\]

        解得:

        \[N=24.868\]

        $N$是整数,于是我们知道$N$的最小值是$25$

        为了实验的准确性,我采用了30次循环的时间来进行实验

\section{实验步骤}

    1. 按照实验要求组装实验仪器,调整水平,不要让球运动的时候触碰到板面,避免对实验造成影响

    2. 将电子秒表示数归零。

    3. 测量摆球的直径 $d$,摆线的长度 $l_{0}$,并计算摆长 $l$。

    4. 将摆球拉离平衡位置,无初速度地释放,使其在小角度(小于$5^\circ $)平面内摆动。
    
    5. 用电子秒表测量单摆 30 次全振动所需要的时间。
    
    6. 重复上述实验操作 6 次,记录有关数据。

    7. 改变绳长,记录不同绳长的实验数据

    8. 对大摆角的过程进行录像,以便使用Tracker进行研究
    
    9. 整理仪器,结束实验。
    
    10. 数据处理和误差分析。

\section{测量记录(原始数据)}

    \begin{table}[!hbtp]
        \begin{center}
        \caption{测量 50 个周期所用时间 $T^{'}$,周期$T$}
        \begin{tabular}{l|c|c|c|c|c|c|r} % <-- Alignments: 1st column left, 2nd middle and 3rd right, with vertical lines in between
            \textbf{parameter$/_{s}$} & \textbf{Value 1} & \textbf{Value 2} & \textbf{Value 3} & \textbf{Value 4} & \textbf{Value 5} & \textbf{Value 6} & \textbf{average}\\
            \hline
            $T^{'}$ & 50.85 & 50.93 & 51.06 & 51.04 & 51.05 & 51.10 & 51.005 \\
            $T$ & 1.695 & 1.697 & 1.702 & 1.7013 & 1.7017 & 1.703 & 1.7002 \\
        \end{tabular}
        \end{center}
    \end{table}

    \begin{table}[!hbtp]
        \begin{center}
        \caption{钢卷尺测量摆球的直径 $d$, 钢卷尺测量摆线的长度 $l_{0}$,并计算摆长 $l$}
        \begin{tabular}{l|c|c|c|c|c|c|r} % <-- Alignments: 1st column left, 2nd middle and 3rd right, with vertical lines in between
            \textbf{parameter$/_{cm}$} & \textbf{Value 1} & \textbf{Value 2} & \textbf{Value 3} & \textbf{Value 4} & \textbf{Value 5} & \textbf{Value 6} & \textbf{average}\\
            \hline
            $d$ & 2.1995 & 2.1995 & 2.1995 & 2.1995 & 2.1995 & 2.1995 & 2.1995 \\
            $l_{0}$ & 70.8 & 70.8 & 70.8 & 70.8 & 70.8 & 70.8 & 70.8 \\
            $l$ & 71.9 & 71.9 & 71.9 & 71.9 & 71.9 & 71.9 & 71.9 \\
        \end{tabular}
        \end{center}
    \end{table}

    \begin{table}[!hbtp]
        \begin{center}
        \caption{摆长$l/_{cm}$和周期$T/_{s}$}
        \begin{tabular}{l|c|c|c|c|c|r} % <-- Alignments: 1st column left, 2nd middle and 3rd right, with vertical lines in between
            \textbf{parameter} & \textbf{Value 1} & \textbf{Value 2} & \textbf{Value 3} & \textbf{Value 4} & \textbf{Value 5} & \textbf{Value 6}\\
            \hline
            $l/_{cm}$ & 71.9 & 64.8 & 57.2 & 51.7 & 74.9 & 76.4 \\
            $T/_{s}$ & 1.700 & 1.617 & 1.518 & 1.438 & 1.744 & 1.752 \\
        \end{tabular}
        \end{center}
    \end{table}

\section{数据处理与误差分析}

    \subsection{不确定度分析}
        摆线长度的平均值:
            \[\overline{l_{0}}=70.8cm\]
        摆线长度的标准差:
            \[\sigma _{l_{0}}=\sqrt{\frac{6(70.8-70.8)^{2}}{6-1}}=0cm\]
        可得展伸不确定度为:
            \[U_{1}=\sqrt{(t_{0.68}\frac{\sigma _{l_{0}}}{\sqrt{n}})^{2}+(k_{p}\frac{\Delta _{r}}{C})^{2}}cm\approx\sqrt{(1.20\times \frac{0}{\sqrt{6}})^{2}+(1.00\times \frac{0.2}{3})^{2}}cm=0.067cm\]
            \[P=0.68\]
        摆球直径的平均值(使用游标卡尺进行测量):
            \[\overline{d}=2.1995cm\]
        摆球直径的标准差:
            \[\sigma _{d}=\sqrt{\frac{6(2.1995-2.1995)^{2}}{6-1}}=0cm\]
        可得展伸不确定度为:
            \[U_{2}=\sqrt{(t_{0.68}\frac{\sigma _{d}}{\sqrt{n}})^{2}+(k_{p}\frac{\Delta _{v}}{C})^{2}}cm\approx\sqrt{(1.20\times \frac{0}{\sqrt{6}})^{2}+(1.00\times \frac{0.002}{3})^{2}}cm=6.7\times 10^{-4}cm\]
            \[P=0.68\]
        综合以上,可得摆长的平均值:
            \[\overline{l}=\overline{l_{0}}+\frac{1}{2}\overline{d}=71.9cm\]
        利用误差传递公式,摆长的展伸不确定度为:
            \[U_{l}=\sqrt{U_{1}^2+(\frac{U_{2}}{2})^{2}}cm\approx 0.067cm\]
        单摆周期的平均值:
            \[\overline{T}=\frac{50.85+50.93+51.06+51.04+51.05+51.10}{30\times 6}\approx1.700s\]
        单摆周期的标准差:
            \[\sigma _{T}=\sqrt{\frac{(\frac{50.85}{30}-1.700)^{2}+(\frac{50.93}{30}-1.700)^{2}+(\frac{51.06}{30}-1.700)^{2}+(\frac{51.04}{30}-1.700)^{2}+(\frac{51.05}{30}-1.700)^{2}+(\frac{51.10}{30}-1.700)^{2}}{6-1}}s\]
            \[\approx 3.166\times10^{-3}s\]
        本实验的测量过程中,B类不确定度:
            \[\Delta _{T}=\frac{1}{30}\sqrt{(\Delta_{person})^{2}+(\Delta_{s})^{2}}=\frac{1}{30}\sqrt{(0.2)^{2}+(0.01)^{2}}s\approx6.675\times10^{-3}s\]
        于是周期 T 的伸展不确定度为:
            \[U_{T}=\sqrt{(t_{0.68}\frac{\sigma _{T}}{\sqrt{n}})^{2}+(k_{p}\frac{\Delta _{T}}{C})^{2}}s=\sqrt{(1.20\times \frac{3.166\times10^{-3}}{\sqrt{6}})^{2}+(1.00\times \frac{6.675\times10^{-3}}{3})^{2}}s\approx 2.71\times10^{-3} s\]
            \[P=0.68\]
        所以本实验测定的重力加速度:
            \[\overline{g}=\frac{4\pi^{2}\overline{l}}{\overline{T}^{2}}=\frac{4\pi ^{2}\times71.9\times10^{-2}}{(1.700)^{2}}\approx 9.822m/s^{2}\]
        $g$的伸展不确定度为:
            \[\frac{U_{g}}{g}=\sqrt{(\frac{U_{l}}{l})^2+4(\frac{U_{T}}{T})^2}=\sqrt{(\frac{0.067}{71.9})^2+4(\frac{2.71\times10^{-3}}{1.700})^2}\approx3.32\times10^{-3}<1\%\]
            \[P=0.68\]
        于是
        \[U_{g}=3.32\times10^{-3}\times9.822=3.26\times10^{-2}m/s^{2},P=0.68\]
        最终的测量结果可以表示为:
        \[g=\overline{g}\pm U_{g}=9.822\pm 0.0326m/s^{2},P=0.68\]
        合肥的重力加速度:$9.7947m/s^{2}$,可以看到实验结果符合实际情况.

    \subsection{最小二乘法拟合$g$}
        这是geogebra绘制的$l$与$T^{2}$的图像

        \begin{figure}[!hbtp]
            \centering
            \includegraphics[width=9.5cm,height=8cm]{./images/拟合.png}
            \caption{拟合图像}
        \end{figure} 

        为了方便$g$的获取,这是$4\pi ^{2}l$与$T^{2}$的回归方程式
        \[4\pi ^{2}l=9.649T^{2}-0.403\]
        \[r=0.999551\]
        \[g=9.649m/s^{2}\]
        
        标准不确定度:$U_{g}=\frac{\Delta g}{g}=\frac{9.7947-9.649}{9.7947}=0.0149=1.49\%$
        
        这里可以看出,使用二次项拟合的方法求重力加速度的误差较大,这其中可能有实验次数较少等因素导致的
    
    \subsection{大摆角($>5^{\circ }$)条件下单摆的运动规律}
        我使用Tracker对视频进行处理,将x方向的偏移进行测量,以下为测量图像:

        \begin{figure}[!hbtp]
            \centering
            \includegraphics[width=9.5cm,height=8cm]{./images/x-t.png}
            \caption{x-t}
        \end{figure}

        我们可以看到这个结果和小角度的时候几乎一致

        为了方便讨论,我们将$x$转换为$\theta_{r}=\arcsin(\frac{x}{l})$进行讨论

        我们讨论小角度和大角度下$\theta$之间的差距,小角度下:
        \[\theta=\theta_{max}sin((T-T_{m})\sqrt{\frac{g}{l}}+\frac{\pi}{2})\]
        \[\Delta_{\theta}=\theta-\theta_{r}\]

        我们将$\Delta_{\theta}$与$T$进行绘图有:

        \begin{figure}[!hbtp]
            \centering
            \includegraphics[width=14cm,height=8cm]{./images/delta-t.png}
            \caption{$\Delta_{\theta}-t$}
        \end{figure}


        可以看到,大角度与小角度运动角度之间差值会相差一个振荡值,
        这个振荡值的周期约为我们的运动周期,
        这个振荡值产生的原因是大摆角会产生较为明显的能量衰减,也就是振幅以指数的形式进行衰减,
        由于时间较短,所以看到的结果与正比例函数一致
        
\section{讨论与思考题}

    \subsection{讨论}
        我在实验过程中采用的是在释放的同时进行计时,然而实际上,在摆球稳定摆动后,当摆球达到最低点时开始计时是比较合适的。
        这样可以避免因观测最高点的困难带来的测量误差,提高实验精度,同时减少初始释放时的不稳定因素对实验结果造成影响。
    \subsection{思考题}
        
    \subsubsection{分析实验测量误差的主要来源,提出可能的改进方案}
        
        1.摆球摆动不在一个平面中进行,可能带来误差。
        
        改进方法:无初速度释放,并调节仪器使得小球在平面内摆动,最好可以不使用手进行释放,防止出现摆动出现与预设平面的偏移
        
        2.计时的方式和时机不准确

        改进方法:在摆球稳定摆动后,当摆球达到最低点时开始计时
\pagebreak
\section{实验记录原始数据}
\begin{figure}[!hbtp]
    % \centering
    \includegraphics[width=8cm,height=12cm]{./images/first_part.jpg}
    \includegraphics[width=8.6cm,height=11.9cm]{./images/second_part.jpg}
    \includegraphics[width=11cm,height=8cm]{./images/big_theta.png}
\end{figure}
\end{document}